\documentclass[12pt,a4paper]{article}

\usepackage[romanian]{babel}
\usepackage[utf8x]{inputenc}
\usepackage{amsmath}
\usepackage{graphicx}
\usepackage{gensymb}
\usepackage[colorinlistoftodos]{todonotes}
\usepackage{combelow}
\usepackage{newunicodechar}

\newunicodechar{Ș}{\cb{S}}
\newunicodechar{ș}{\cb{s}}
\newunicodechar{Ț}{\cb{T}}
\newunicodechar{ț}{\cb{t}}
\newunicodechar{Ă}{\cb{A}}
\newunicodechar{ă}{\cb{a}}
\newunicodechar{Â}{\cb{A}}
\newunicodechar{â}{\cb{a}}
\newunicodechar{Î}{\cb{I}}
\newunicodechar{î}{\cb{i}}

\title{Inside NAV}
\author{Barcan Virgil-Gheorghe}

\begin{document}
\maketitle


\section{Sistemul de operare Android}
\subsection{Despre Android}
Android este un sistem de operare dedicat dispozitivelor mobile, smartphone-uri, smartwatch-uri, precum și TV-urilor sau chiar dispozitive din interiorul automobilelor.

Sistemul de operare Android este bazat pe un kernel de Linux și a fost construit pentru a fi folosit în principal pe dispozitive touchscreen. Este cel mai folosit sistem de operare disponibil pe dispozitive mobile încă din 2013.

Sistemul a fost dezvoltat de către Android Inc., companie fondată în 2003. Dezvoltatorii au avut dorința de a ajuta la construirea de dispozitive mai inteligente, conștiente de locația și preferințele utilizatorilor.

În 2005 compania Android Inc. a fost achiziționată de către Google pentru o sumă de aproximativ 50 milioane de dolari, iar intenția Google era clară pentru mulți, intrarea pe piața dispozitivelor mobile.

La data de 5 noiembrie 2007, un consorțiu de companii precum Google, HTC, Sony, Samsung, T-Mobile, dar și producători de chipset-uri, Qualcomm și Texas Instruments, a fost lansat. Acest consorțiu, numit „Open Handset Alliance”, al cărui scop era de a dezvolta standarde deschise pentru dispozitive mobile, a lansat în aceeași zi sistemul de operare Android, ca prim produs al său.

Primul smartphone cu sistem de operare Android lansat a fost HTC Dream, în luna octombrie a anului 2008.

Fiecare nouă versiune a sistemului este denumită în ordine alfabetică după un desert: „Cupcake”, „Donut”, „Eclair”, „Froyo”, „Gingerbread”, „Honeycomb”, „Ice Cream Sandwich”, „Jellybean”, „Kit Kat”, „Lolipop” și „Marshmallow”. Următoarea versiune este plănuită pentru acest an, deocamdată ea fiind denumită Android N. Google a oferit utilizatorilor posibilitatea să voteze numele.


\subsection{Versiunile Android}
\subsubsection{Android 1.0}
Prima versiune a sistemului de operare Android, denumită, deloc pompos, Android 1.0, dădea impresia de produs nefinalizat, însă lăsa să se întrevadă planurile Google pentru această platformă. Această versiune deja avea părți care, comparate cu ce era  pe piață la momentul acela, erau mult superioare. Sunt lucruri pe care acum le trecem cu vederea, dar care atunci erau noutăți, widget-urile, zonele de notificări și nu numai. 

Prima versiune a Android includea și aplicațiile cu care suntem acum atât de obișnuiți, cum ar fi Gmail sau YouTube. Android Market Beta debuta și ea, oferind posibilitatea de a lista aplicații și jocuri.

Tot în acea perioadă Android și-a câștigat și atât de cunoscutul logo, denumit în interiorul Google „Bugdroid”.


\subsubsection{Android 1.5 - Cupcake}
Primul mare update al Android-ului a fost versiunea 1.5, denumită, în ceea ce avea să stabilească trendul numelor inspirate din dulciuri, „Cupcake”.

	Lansat în aprilie 2009, Cupcake a fost deschizător de drum pentru dispozitivele cu Android fără tastatură fizică. Telefoanele precedente aveau încorporate și tastaturi fizice, însă cu această versiune de sistem de operare, Google a introdus tastatura virtuală, precum și posibilitatea folosirii de tastaturi third-party.

	Acestea arătau evenimentele și respectiv melodia curentă. Pentru prima dată aplicațiile puteau să își adauge widget-uri proprii. O altă noutate a fost reprezentată de posibilitatea de a roti automat ecranul în funcție de orientarea telefonului, pentru o tranziție ușoară portret-landscape.


\subsubsection{Android 1.6 - Donut}
În același an al lansării Android 1.5, Google a lansat și noua versiune, 1.6, numită „Donut”.
	
	Aceasta aducea suport pentru ecrane de diferite rezoluții și densități ale pixelilor, precum și suport nativ pentru comunicarea în banda CDMA (Code Division Multiple Access). Noul sistem oferea și posibilitatea de a efectua căutări, atât în fișierele aflate pe telefon, cât și pe internet, prin așa numitul „Quick Search Box”. Tot cu această versiune a venit și posibilitatea de a vedea consumul de baterie al dispozitivului.
	
	Android Market a fost refăcut pentru a putea expune aplicațiile gratuite de top, precum și aplicațiile de top plătite, în contextul exploziei de noi aplicații de pe această platformă. Pentru prima dată au fost introduse și butoane care ofereau acces facil la setări precum Wi-Fi, Bluetooth, GPS, Sincronizare sau Luminozitate.


\subsubsection{Android 2.0, 2.0.1, 2.1 - Eclair}
Lansat în octombrie 2009, Android 2.1, denumit și „Eclair”, a adus câteva îmbunătățiri precum: posibilitatea de a căuta în toate SMS-urile și MMS-urile salvate pe telefon, viteză de tastare mărită pentru tastatura virtuală, setări de accesibilitate, calendar, dar și un API pentru VPN (Virtual Private Network). Pentru browser, Android aducea suport HTML5. Tot acum au apărut double-tap-zoom și pinch-to-zoom.
	
	Tot în versiunea Eclair a fost introdusă și posibilitatea de a folosi Google Maps pentru navigație, aceasta oferind indicații pas-cu-pas, dar și informații legate de trafic, caracteristici similare serviciilor de navigație din automobile, cu diferența că cele oferite de Google erau gratuite.

	O altă funcționalitate care acum este extrem de utilizată și a fost adăugată înca din versiunea Eclair este cea de Speech-to-Text, care permitea introducerea de text dictând telefonului.
	

\subsubsection{Android 2.2 - Froyo}
Poate cea mai importantă noutate a acestei versiuni a fost aceea a introducerii mașinii virtuale Dalvik, care oferea Just-in-time Compiling, oferind îmbunătățiri masive ale vitezei Android-ului.

	Cu această versiune a fost oferită și posibilitatea de a instala sau muta aplicațiile pe cardul de memorie pentru a elibera memoria internă a telefonului. Altă noutate a fost reprezentată de hotspot-ul Wi-Fi, care permitea telefonului să ofere Wi-Fi altor dispozitive.

	Browser-ul a cunoscut și el îmbunătățiri ale vitezei prin schimbarea motorului de JavaScript la cel folosit de Chrome, V8.


\subsubsection{Android 2.3, 2.3.3 - Gingerbread}
Cu această nouă versiune a Android a apărut și ceea ce avea să devină un nou obicei al Google, acela de a lansa un telefon sub marcă proprie la câteva versiuni de soft distanță. Aceasă linie de telefoane se va numi Nexus și va fi construită cu producători diferiți la fiecare nouă iterație, ea oferind un Android „curat”. Primul telefon din gama Nexus a fost Nexus S, oferit de către Google în parteneriat cu Samsung.

	Sistemul de operare a adus suport pentru NFC (Near Field Communication), pentru ecrane cu rezoluții mari, pentru telefonie via internet (VoIP), dar și îmbunătățiri ale interfeței vizuale.

	Noul garbage collector concurent a adus un spor de viteză. A fost mărită și viteza de distribuție a evenimentelor în sistem. Tot cu această versiune s-a adăugat și suport nativ pentru mai mulți senzori (cum ar fi giroscopul și barometrul), suport pentru multiple camere video, dar și posibilitatea efectuării de apeluri video.

	Pentru dezvoltatori a fost îmbunătățit suportul pentru scrierea de cod nativ și s-au introdus API-urile pentru jocuri.


\subsubsection{Android 3.0 - Honeycomb}
Această nouă versiune a fost lansată cu scopul de a îmbunătăți experiența de utilizare a sistemului pe tablete. Pe lângă îmbunătățirile aduse la nivelul interfeței grafice, au fost aduse și îmbunătățiri la nivelul suportului pentru procesoare multi-core.

	A fost introdus și API-ul pentru utilizarea Fragmentelor, care permiteau dezvoltatorilor de aplicații sa folosească suprafața mai mare a ecranului unei tablete pentru a afișa mai multe părți ale aplicației. Folosirea Fragmentelor permitea separarea aplicațiilor pentru telefoane de cele pentru tablete într-un mod foarte simplu: dacă exista suficient spațiu, se afișau mai multe fragmente ale aplicației, altfel se afișa doar unul și utilizatorul trebuia să navigheze între acestea.

	A fost introdus și conceptul de Action Bar, adică acel spațiu din partea superioară a unei aplicații similar unui meniu dintr-o aplicație desktop.


\subsubsection{Android 4.0 - Ice Creak Sandwich}
Această versiune oferea o experiență unificată smartphone-tabletă, ea înlocuind complet versiunea Honeycomb și aducând noutățile ei și pe smartphone-uri.

	A fost îmbunătățit multitaskingul, notificările au devenit mai bogate, ecranul de start au fost și el modificat pentru a se reduce consumul de spațiu. S-a adăugat un sistem de management al consumului de date.

	Multitaskingul a devenit acum mai vizibil pentru utlizatorul final, acesta putând acum să selecteze aplicația pe care să o redeschidă. Utilizatorul poate vedea o listă de aplicații pe care le-a folosit în ultima perioadă, apoi poate selecta aplicația pe care să o readucă în prim-plan.

	În partea de jos a ecranului au apărut aplicațiile favorite, într-un aranjament similar celui de pe iPhone. Gesturile încep să devină parte esențială a experienței de utilizare a smartphone-ului, ele putând fi folosite pentru a șterge notificări, pentru a schimba tab-urile în browser, dar și pentru a respinge/accepta apeluri.

	Una din inovațiile aduse a fost Android Beam. Acest sistem permitea transferul de date între dispozitive care dispuneau de NFC. Acest transfer se făcea cu viteze mai mari decât ale Bluetooth-ului, într-un mod extrem de facil din perspectiva utilizatorului. Acesta trebuia doar să apropie cele două dispozitive și să apese „Trimitere”.


\subsubsection{Android 4.1, 4.2, 4.3 - Jelly Bean}
Această iterație a Android a adus noi îmbunătățiri la capitolul viteză prin introducerea „Project Butter”, care implementa Vsync și triple-buffering. Notificările au fost și ele modificate, putând conține acum până la 8 linii de text și chiar butoane la baza lor. Aceste butoane puteau fi folosite pentru a lua acțiuni în funcție de notificare.

	În Android Jelly Bean a fost lansat și Google Now, asistentul virtual al Google. Acesta oferă informații legate de vremea din locația curentă, precum și despre timpii necesari pentru a ajunge între diferite locații.

	Printre noutăți se numără și posiblitatea de a avea mai multe conturi pe același dispozitiv. Tot acum s-a adăugat și suportul pentru ecrane externe, eventual via Miracast (Wi-Fi Display).

	Alte îmbunătățiri sunt date de adăugarea HDR (High Dynamic Range), a Bluetooth Low Energy, dar și a OpenGL ES3.0, care aducea sporuri de performanță la nivel grafic.


\subsubsection{Android 4.4 - KitKat}
Poate cea mai importantă funcționalitate adăugată cu această versiune este cea a interacțiunii cu telefonul prin comenzi vocale. Faimosul „Ok, Google” a fost oferit începând cu această versiune. Tot acum s-au făcut eforturi pentru ca Android să poată rula mai bine pe dispozitive mai slabe din punct de vedere hardware. Acum Android putea rula chiar și pe 512 MB de RAM.

	S-a adăugat și un nou sistem de acces la stocare, care a permis dezvoltatorilor precum Box sau Dropbox să integreze serviciile lor direct cu memoria telefonului, oferind acces facil la documente din locații diferite.

	Au fost imbunătățiți și senzorii, prin reducerea consumului lor de baterie. Android nu  mai trimite notificările imediat cum senzorii observă modificările, ci le stochează până are suficiente.  Acum au fost introduși și senzorii de detectare a pașilor și de numărare a pașilor. 
	

\subsubsection{Android 5.0, 5.1 - Lollipop}
Android Lollipop a fost versiunea în care a apărut noul stil vizual propus de Google, „Material Design”. Acest ghid despre cum trebuie să arate interfața vizuală a aplicației a fost un mare pas către unificarea experienței de utilizare a aplicațiilor de pe această platformă.
	
	Material Design este un set de reguli de design inspirat din modul în care arată hârtia în diferite combinații de lumini și umbre.
Componentele grafice din aplicații erau acum reprezentate într-un mod minimalist, ca și cum ar fi niște colaje de hârtie. Google nu a adus doar recomandările legate de cum trebuie să arate componentele grafice, ci și un întreg set de instrumente noi care să ajute dezvoltatorii.
	
	Android 5.0 a adus și îmbunătățiri de performanță prin adăugarea ART runtime în locul Dalvik. Acum există suport pentru procesoare cu arhitectura pe 64 de biți. S-a adăugat suport pentru OpenGL ES 3.1, care a dus la jocuri mai bogate grafic și mai captivante.
	
	Tot începând cu această versiune Android și-a făcut apariția și în zona TV, prin lansarea Android TV. Android nu s-a oprit însă la zona TV, el apărând și în zona auto, prin Android Auto.

	
\subsubsection{Android 6.0 - Marshmallow}
Android Marshmallow a introdus un nou sistem de permisiuni pentru aplicații. Acest nou model presupunea ca aplicația să ceară drepturi de la utilizator nu la instalare, ci atunci când îi era necesar. Prin acest nou set de reguli se dorea ca utilizatorii să știe mai clar ce acceptă, deoarece înainte utilizatorii nu erau neapărat atenți la ce acceptau, și puteau exista aplicații care să poată avea acces la mai multe date decât le era necesar. Existau aplicații de genul „Lanternă” care aveau acces la contacte, locație, microfon și cameră. Genul acesta de permisiuni sunt în mod clar un pericol la intimitatea utilizatorilor, iar cu Android 6.0 Google a încercat să îl limiteze.

	O altă componentă a sistemului unde cei de la Google au adus îmbunătățiri este cea a consumului de baterie. Prin adăugarea „Doze” și a „App Standby”, aplicațiile consumă acum mai puțină baterie.

	În Marshmallow s-a adăugat și Now On Tap, un asistent virtual care permite să afli informații fără să părăsești aplicația curentă. Este util atunci când vrei să afli ce reprezintă ceva din aplicația curentă.


\subsubsection{Android N}
O nouă versiune a Android este în dezvoltare acum, aceasta având numele de cod „Android N”. Un nume care să urmeze linia numelor inspirate din produse dulci este în curs de a fi ales, Google oferind utilizatorilor posibilitatea să recomande și să voteze numele.
	
	Versiunea Android N este acum în developer preview. Aceast Android promite îmbunătățiri în 3 zone cheie: performanță, productivitate și securitate.
	
	Modificările nu se vor opri aici, ci vor include, pentru prima dată în Android, suport pentru mai multe ferestre. Android Instant Apps va permite testarea unor aplicații fără ca acestea să fie instalate efectiv pe dispozitiv.
	

\subsection{Senzorii disponibili pe platforma Android}
Majoritatea dispozitivelor care rulează Android au încorporați diverși senzori. Aceștia sunt folosiți în diferite contexte în timpul folosirii dispozitivului. Scopul senzorilor este să urmărească eventuale schimbări astfel încât sistemul să răspundă în cel mai bun mod cu putință.

	Deși poate nu este evident, impactul senzorilor este major. Faptul că ecranul se închide atunci când apropiem telefonul pentru a vorbi, faptul că putem să ne măsurăm distanțele parcurse, bătăile inimii, faptul că putem să folosim telefonul ca sistem de navigație, toate acestea se datorează folosirii senzorilor.
	
	Android împarte senzorii în mai multe categorii, deși tratarea lor se face într-un mod aproape unitar: senzori de \textbf{mișcare}, de \textbf{mediu}, de \textbf{poziție} și de \textbf{locație}.

	Încă un mod în care Android împarte senzorii este acela că senzorii pot fi \textbf{hardware} sau \textbf{software}.\\
	
\subsubsection{Sistemul de coordonate al senzorilor}
În general Android folosește un sistem de coordonate standard, cu 3 axe. Valorile venite de la senzori reprezintă schimbările pe aceste trei axe.

Sistemul de coordonate este definit în raport cu ecranul dispozitivului (ținut în pozitie verticală), astfel: axa $X$ este orizontală și orientată către dreapta, axa $Y$ este verticală si orientată către partea superioară a dispozitivului, iar axa $Z$ este axa care „iese” din centrul ecranului.

\begin{figure}[hbtp]
\centering
\includegraphics[width=5cm]{figures/axis_device.png}
\caption{Sistemul de coordonate}
\end{figure}


\subsubsection{Senzori de mișcare}
Acești senzori oferă posibilitatea monitorizării mișcării dispozitivului, de exemplu rotație, scuturare, înclinare sau legănare. Doi dintre acești senzori sunt mereu senzori fizici: \textbf{accelerometrul} și \textbf{giroscopul}. Ceilalți trei senzori pot fi atât hardware, cât și software: senzorul de \textbf{gravitație}, cel de \textbf{accelerație liniară} și \textbf{senzorul vectorului de rotație}. Recent au fost adăugați și senzori de \textbf{detectat}, respectiv \textbf{numărat pași}.

	\textbf{Accelerometrul} determină accelerația aplicată dispozitivului prin măsurarea forțelor aplicate senzorului în sine. Totuși, în aceste măsurători intră mereu în calcul gravitația.

	\textbf{Senzorul de gravitație} oferă un vector 3-dimensional care indică direcția și magnitudinea gravitației.

	\textbf{Senzorul de accelerație liniară} oferă un vector 3-dimensional care indică accelerația pe fiecare axă, excluând gravitația.

	Putem formaliza relația dintre accelerații astfel:
	$A_{liniara} = A - g$.\\

	\textbf{Giroscopul} măsoară viteza angulară în raport cu axele $X$, $Y$, $Z$. Viteza angulară este măsurată în $rad/sec$.\\

	\textbf{Senzorul vectorului de rotație} reprezintă orientarea dispozitivului ca fiind combinația dintre o axă și un unghi. Dispozitivul este rotit în jurul uneia dintre axele $X$, $Y$, $Z$ cu un unghi $\theta$. Cele trei componente ale vectorului de rotație sunt: $(x \sin{\theta}, y \sin{\theta}, z \sin{\theta})$. 
	
	Comparativ cu ceilalți senzori de mișcare, acest senzor are valorile exprimate într-un alt sistem de coordonate, unul „global”.
	Acest sistem are următoarele caracteristici:
	\begin{itemize}
	\item $X$ este definit ca fiind tangent la pământ în locația curentă și orientat către Est,
	\item $Y$ este definit ca fiind tangent la pământ în locația curentă și orientat către Polul Nord Magnetic,
	\item $Z$ este definit ca fiind perpendicular la planul definit de $X$ și $Y$, orientat către cer.
	\end{itemize}

\begin{figure}[h]
\centering
\includegraphics[width=5cm]{figures/axis_globe.png}
\caption{Sistemul de coordonate utilizat de senzorul vector de rotație}
\end{figure}

\subsubsection{Senzori de mediu}
Acești senzori permit monitorizarea unor valori precum \textbf{umiditatea}, \textbf{intensitatea luminoasă}, \textbf{presiunea atmosferică} sau \textbf{temperatura mediului} din apropierea dispozitivului. Toți senzorii menționați sunt hardware.

Senzorii de \textbf{intensitate luminoasă}, \textbf{presiune atmosferică} sau \textbf{temperatură} sunt unii dintre cei mai simplu de utilizat deoarece nu este necesar ca datele lor să fie corectate. Ei oferă valori cu următoarele unități de măsură: $lx$ (lux) pentru intensitatea luminoasă, $mbar$ (milibar) pentru presiunea atmosferică, respectiv $\celsius$ (grade Celsius) pentru temperatură.

Senzorul de \textbf{umiditate} este și el simplu de folosit. Valorile oferite de el sunt procente.


\subsubsection{Senzori de poziție}
Acești senzori permit monitorizarea poziției dispozitivului într-un cadru de referință global, prin urmărirea schimbărilor de \textbf{câmp magnetic} sau a \textbf{orientării}. Un alt senzor care face parte din această categorie este și  \textbf{senzorul de proximitate}.

\textbf{Senzorul de câmp magnetic} arată fluctuațiile câmpului magnetic al Pământului pe cele trei axe. El oferă valori a căror unitate de măsură este $\mu$T (microTesla). În mod uzual datele de la acest senzor nu sunt folosite în forma lor brută, ci în combinație cu alte date. De exemplu, datele sale, împreună cu cele venite de la accelerometru, pot fi folosite pentru a obține matricile de rotație și de înclinație. Aceste matrici pot fi foloste pentru a se obține \textbf{azimutul}. Azimutul este unghiul la care suntem față de Polul Nord Magnetic. După declinarea magnetică acest unghi ajunge să arate diferența față de Polul Nord, acționând similar unei busole.

\textbf{Senzorul de orientare} permite monitorizarea poziției relative a dispozitivului în raport cu sistemul de coordonate global. Acest senzor face fuziunea descrisă mai sus, combinând datele de la magnetometru și de la accelerometru pentru a obține \textbf{azimutul} (gradele de rotație în jurul axei $Z$), \textbf{pitch}-ul (gradele de rotație în jurul axei $X$) și \textbf{roll}-ul (gradele de rotație în jurul axei $Y$). Deoarece combinarea datelor de la ceilalți doi senzori consumă foarte mult, precizia senzorului este diminuată. De aceea, acest senzor a fost declarat depășit și Android recomandă utilizarea metodei prezentate mai sus pentru obținerea orientării (via matricile de rotație și înclinație).

\textbf{Senzorul de proximitate} raportează cand telefonul este „aproape” sau „departe” de un alt obiect. În mod uzual distanța începând cu care un obiect este considerat apropiat/depărtat este de 5 $cm$.

\subsubsection{Senzori de locație}
Acești senori primesc date despre locația curentă a dispozitivului. Locația curentă poate fi obținută din mai multe surse, cu erori mai mici sau mai mari. Printre surse numim: \textbf{GPS}-ul, \textbf{Wi-Fi}-ul sau chiar datele de \textbf{celulă telefonică}. Pe lângă erorile ce pot veni din faptul că avem de a face cu multiple surse, nici datele provenite din aceeași sursă nu sunt cele mai precise.

Android oferă aplicațiilor acces la servicii de localizare în funcție de ce este disponibil pe dispozitiv. API-ul pentru locație este puțin diferit față de cel pentru ceilalți senzori.\\

Deși acești senzori sunt recunoscuți de către Android, nu este obligatoriu ca toți să fie prezenți și pe dispozitiv. Android nu impune ce senzori să fie prezenți pe dispozitive, ei doar recomandă prezența unora de bază pentru o utilizare cât mai apropiată de cea normală. Producătorii de dispozitive pot la fel de bine să mai adauge senzori noi, nici acest lucru nu este îngrădit de modul de lucru al Android. Fiind open-source, i se poate adăuga suport destul de ușor pentru senzori noi.





\end{document}